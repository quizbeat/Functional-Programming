\documentclass[a4paper, 12pt]{article}

\usepackage{fullpage}
\usepackage{multicol,multirow}
\usepackage{tabularx}
\usepackage{ulem}
\usepackage[utf8]{inputenc}
\usepackage[russian]{babel}
\usepackage{amsmath}
\usepackage{amssymb}
\usepackage{listings}
\usepackage{titlesec}

\titleformat{\section}
  {\normalfont\Large\bfseries}{\thesection.}{0.3em}{}

\titleformat{\subsection}
  {\normalfont\large\bfseries}{\thesubsection.}{0.3em}{}

\titlespacing{\section}{0pt}{*2}{*2}
\titlespacing{\subsection}{0pt}{*1}{*1}
\titlespacing{\subsubsection}{0pt}{*0}{*0}
\usepackage{listings}
\lstloadlanguages{Lisp}
\lstset{extendedchars=false,
  breaklines=true,
  breakatwhitespace=true,
  keepspaces = true,
  tabsize=2
}

\begin{document}

\section*{Отчет по лабораторной работе №\,5
\\по курсу \guillemotleft  Функциональное программирование\guillemotright}

\begin{flushright}
Студент группы 8о-306Б МАИ \textit{Макаров Никита}, \textnumero 11 по списку \\
\makebox[7cm]{Контакты: {\tt quizbeat@gmail.com} \hfill} \\
\makebox[7cm]{Работа выполнена: 22.05.2015 \hfill} \\
\ \\
Преподаватель: Иванов Дмитрий Анатольевич, доц. каф. 806 \\
\makebox[7cm]{Отчет сдан: \hfill} \\
\makebox[7cm]{Итоговая оценка: \hfill} \\
\makebox[7cm]{Подпись преподавателя: \hfill} \\

\end{flushright}


\section{Тема работы}
Знаки, строки и символы.

\section{Цель работы}
Научиться работать с литерами (знаками) и строками при помощи функций обработки строк и общих функций работы с последовательностями, использовать свойства символов.

\section{Задание}
Запрограммировать на языке Коммон Лисп функцию, принимающую один аргумент - дерево, т.е. список с подсписками, представляющий форму арифметического выражения Lisp. В выражении допустимы только:
\begin{itemize}
\item четыре арифметические функции {\tt +}, {\tt -}, {\tt *} и {\tt /} (предусмотреть случай "унарных"\;функций {\tt -} и {\tt /},
\item символы переменных,
\item числовые константы.
\end{itemize}
Функция должна вернуть строку этого арифметического выражения в постфиксной польской записи.

\begin{lstlisting}
(form-to-postfix '(+ (* b b) (- (* 4 a c)))) => "b b * 4 a c * +"
\end{lstlisting}

\section{Оборудование студента}
Процессор Intel Core i5 2\,@\,1.3GHz, память: 4Gb, разрядность системы: 64.


\section{Программное обеспечение}
ОС Mac OS X 10.9, среда Clozure CL 1.10


\section{Идея, метод, алгоритм}
Небольшое замечание: результат работы функции в примере задания не является корректным выражением в обратной польской записи, так как, во-первых, упущен унарный минус, а во-вторых, недостаточно операторов для обработки выражения.

{\bf Обратная польская нотация} — форма записи математических выражений, в которой операнды расположены перед знаками операций. Также именуется как обратная польская запись, обратная бесскобочная запись, постфиксная нотация и т.п.

Так как требуется перевести арифметическое Lisp-выражение, то будем обрабатывать последовательности вида {\tt (оператор список-операндов)}. В случае, когда длина списка операндов равна 1, оператор является унарным и он обрабатывается довольно просто. Для оператора {\tt /} создается строка вида {\tt ''1 операнд /''}. Для унарного {\tt - } добавляется {\tt 0}, чтобы выразить отрицание и получается строка вида {\tt ''0 операнд -''}. Для обработки нескольких операндов функция {\tt proc-args} рекурсивно создает строку, состоящую из приведенного к постфиксной записи первого элемента списка операндов и результата вызова этой же функции от хвоста списка операндов, добавляя на каждом вызове текущий оператор. Все операнды приводятся к постфиксному виду. Функция {\tt single-arg} проверяет, является ли текущий оператор унарным.


\section{Сценарий выполнения работы}


\section{Распечатка программы и её результаты}

\subsection{Исходный код}
\begin{lstlisting}
(defun single-arg (expr) ;; checks for unary operator
  (and (not (null expr)) (not (null (rest expr))) (null (rest (rest expr))))
)

(defun form-to-postfix (expr) 
  (if (listp expr)
      (if (single-arg expr)
          (if (eq (first expr) '- )
              (concatenate 'string "0 "(princ-to-string (form-to-postfix (second expr))) " -")
              (concatenate 'string "1 " (princ-to-string (form-to-postfix (second expr))) " /")
          )
          (concatenate 'string (form-to-postfix (second expr)) " "
                               (proc-args (first expr) (rest (rest expr))))
      )
      (string-downcase (princ-to-string expr))
  )
)

(defun proc-args (operator expr) ;; processes the given arguments
  (if (not (null expr))
      (if (null (rest expr))
          (concatenate 'string (form-to-postfix (first expr)) " " 
                               (princ-to-string operator)) 
          (concatenate 'string (form-to-postfix (first expr)) " " 
                               (princ-to-string operator) " " 
                               (proc-args operator (rest expr)))
      )
  )
)

(defun t (expr) (form-to-postfix expr))
\end{lstlisting}


\subsection{Результаты работы}
\begin{lstlisting}
> (form-to-postfix '(+ (*  b b) (- (* 4 a c))))
"b b * 0 4 a * c * - +"
> (form-to-postfix '(* (- 2) (/ 3)))
"0 2 - 1 3 / *"
> (form-to-postfix '(+ 7 (* 2 3)))
"7 2 3 * +"
> (form-to-postfix '(* (+ (- 2) (* 5 6)) (/ (- 9 3))))
"0 2 - 5 6 * + 1 9 3 - / *"
> (form-to-postfix '(+ (- (* (/ a) 2 c 4) e 5 g) 7 i))
"1 a / 2 * c * 4 * e - 5 - g - 7 + i +"
\end{lstlisting}


\section{Дневник отладки}

\section{Замечания, выводы}
В языке Common Lisp есть неплохие средства для работы со строками и символами, такие как конкатенация, удаление заданных символов, проверка совпадений, поиск и прочие. Применив рекурсию и некоторые из упомянутых выше функции можно легко реализовать программу, преобразовывающую арифметическое выражение в какую-либо нотацию. В данной лабораторной работе производится преобразование в обратную польскую нотацию, которая немного напоминает запись Lisp-выражений, но располагает операторы справа от операндов.

\end{document}