\documentclass[a4paper, 12pt]{article}

\usepackage{fullpage}
\usepackage{multicol,multirow}
\usepackage{tabularx}
\usepackage{ulem}
\usepackage[utf8]{inputenc}
\usepackage[russian]{babel}
\usepackage{amsmath}
\usepackage{amssymb}
\usepackage{listings}
\usepackage{titlesec}
\usepackage{listings}

\titleformat{\section}
  {\normalfont\Large\bfseries}{\thesection.}{0.3em}{}

\titleformat{\subsection}
  {\normalfont\large\bfseries}{\thesubsection.}{0.3em}{}

\titlespacing{\section}{0pt}{*2}{*2}
\titlespacing{\subsection}{0pt}{*1}{*1}
\titlespacing{\subsubsection}{0pt}{*0}{*0}

\lstloadlanguages{Lisp}
\lstset{extendedchars=false,
	breaklines=true,
	breakatwhitespace=true,
	keepspaces = true,
	tabsize=2
}
\begin{document}

\section*{Отчет по лабораторной работе №\,1 
\\по курсу \guillemotleft  Функциональное программирование\guillemotright}

\begin{flushright}
Студент группы 8о-306Б МАИ \textit{Макаров Никита}, \textnumero 11 по списку \\
\makebox[7cm]{Контакты: {\tt quizbeat@gmail.com} \hfill} \\
\makebox[7cm]{Работа выполнена: 19.03.2014 \hfill} \\
\ \\
Преподаватель: Иванов Дмитрий Анатольевич, доц. каф. 806 \\
\makebox[7cm]{Отчет сдан: \hfill} \\
\makebox[7cm]{Итоговая оценка: \hfill} \\
\makebox[7cm]{Подпись преподавателя: \hfill} \\

\end{flushright}


\section{Тема работы}
Примитивные функции и особые операторы Common Lisp.


\section{Цель работы}
Научиться вводить S-выражения в Lisp-систему, определять переменные и функции, работать с условными операторами, работать с числами, использую схему линейной и древовидной рекурсии.


\section{Задание}
\textit{Золотое сечение} - это число \textit{r}, которое удовлетворяет уравнению
\(r^{2} = r + 1\). \\Покажите, что золотое сечение \textit{r} есть неподвижная точка трансформации\\ \(f(x) = 1 + \frac{1}{x}\), запрограммировав на языке Common Lisp функцию для вычисления \textit{r} с использованием функционала {\tt fixed-point}.


\section{Оборудование студента}
Процессор Intel Core i5 2\,@\,1.3GHz, память: 4Gb, разрядность системы: 64.


\section{Программное обеспечение}
ОС Mac OS X 10.9, среда Clozure CL 1.10


\section{Идея, метод, алгоритм}
Функция {\tt golden-ratio} с помощью функционала {\tt fixed-point} находит золотое сечение \textit{r}. В {\tt fixed-point} объявляются локальная функция {\tt close-enough-p} и локальная рекурсивная функция {\tt try}, которая проверяет, достаточно ли \(f(x)\) приблизилось к \(x\). Если достаточно, то получен ответ, иначе берется следующее приближение.


\section{Сценарий выполнения работы}


\section{Распечатка программы и её результаты}

\subsection{Исходный код}
\begin{lstlisting}
(defvar tolerance 0.00001)
(defconstant start 6.9)

(defun func (x)
    (+ 1 (/ 1 x))
)

(defun fixed-point (f first-guess)
    (labels (
        (close-enough-p (x y)
            (<= (abs (- x y)) tolerance))
        (try (guess)
            (let ((next (funcall f guess)))
                (if (close-enough-p guess next)
                    next
                    (try next)
                )
            )
        ))
        (try first-guess)
    )
)

(defun golden-ratio ()
    (funcall #'fixed-point #'func start)
)
\end{lstlisting}

\subsection{Результаты работы}
\begin{lstlisting}
(golden-ratio)
1.6180363
\end{lstlisting}

\section{Дневник отладки}
\begin{tabular}{|c|c|c|c|}
\hline
Дата & Событие & Действие по исправлению & Примечание \\
\hline
\end{tabular}


\section{Замечания, выводы}
Я получил базовые навыки программирования на функциональном языке Common Lisp, написав несложную программу, вычисляющую золотое сечение. Независимо от начального приближения, программа корректно вычисляет константу \textit{r}.

\end{document}