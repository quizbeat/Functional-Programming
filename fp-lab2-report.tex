\documentclass[a4paper, 12pt]{article}

\usepackage{fullpage}
\usepackage{multicol,multirow}
\usepackage{tabularx}
\usepackage{ulem}
\usepackage[utf8]{inputenc}
\usepackage[russian]{babel}
\usepackage{amsmath}
\usepackage{amssymb}
\usepackage{listings}
\usepackage{titlesec}

\titleformat{\section}
  {\normalfont\Large\bfseries}{\thesection.}{0.3em}{}

\titleformat{\subsection}
  {\normalfont\large\bfseries}{\thesubsection.}{0.3em}{}

\titlespacing{\section}{0pt}{*2}{*2}
\titlespacing{\subsection}{0pt}{*1}{*1}
\titlespacing{\subsubsection}{0pt}{*0}{*0}
\usepackage{listings}
\lstloadlanguages{Lisp}
\lstset{extendedchars=false,
	breaklines=true,
	breakatwhitespace=true,
	keepspaces = true,
	tabsize=2
}

\begin{document}

\section*{Отчет по лабораторной работе №\,2 
\\по курсу \guillemotleft  Функциональное программирование\guillemotright}

\begin{flushright}
Студент группы 8о-306Б МАИ \textit{Макаров Никита}, \textnumero 11 по списку \\
\makebox[7cm]{Контакты: {\tt quizbeat@gmail.com} \hfill} \\
\makebox[7cm]{Работа выполнена: 03.04.2015 \hfill} \\
\ \\
Преподаватель: Иванов Дмитрий Анатольевич, доц. каф. 806 \\
\makebox[7cm]{Отчет сдан: \hfill} \\
\makebox[7cm]{Итоговая оценка: \hfill} \\
\makebox[7cm]{Подпись преподавателя: \hfill} \\

\end{flushright}


\section{Тема работы}
Простейшие функции работы со списками Common Lisp.


\section{Цель работы}
Научиться конструировать списки, находить элемент в списке, использовать схему линейной и древовидной рекурсии для обхода и реконструкции плоских списков и деревьев.

\section{Задание}
Запрограммируйте функцию-предикат {\tt tree-similar-p (x y)}, которая принимает два аргумента - дерева, представленных в виде списков атомов. Предикат должен вернуть истину, если одинаковые атомы расположены в списках {\tt х} и {\tt у} в одном и том же порядке при обходе дерева слева направо, т.е. независимо от внутренней структуры {\tt х} и {\tt у}.

\begin{lstlisting}
(tree-similar-p '(1 (2 (3 4)) 5)) '((1 2) 3 (4 5))) => T
\end{lstlisting}


\section{Оборудование студента}
Процессор Intel Core i5 2\,@\,1.3GHz, память: 4Gb, разрядность системы: 64.


\section{Программное обеспечение}
ОС Mac OS X 10.9, среда Clozure CL 1.10


\section{Идея, метод, алгоритм}
Так как в задании не нужно учитывать структуру дерева, то можно заметить, что достаточно проверить на эквивалентность списки, представляющие деревья. Но так как списки могут быть вложенными, сначала необходимо избавиться от вложенности. Функция {\tt to-list} раскрывает все вложенные списки, а функция {\tt equal} проверяет эквивалентность двух списков.


\section{Сценарий выполнения работы}


\section{Распечатка программы и её результаты}

\subsection{Исходный код}
\begin{lstlisting}
(defun to-list (tree)
    (if (null tree)
        nil
        (if (atom (first tree))
            (cons (first tree) (to-list (rest tree)))
            (append (to-list (first tree)) (to-list (rest tree)))
        )
    )
)

(defun tree-similar-p (tree1 tree2)
    (equal (to-list tree1) (to-list tree2))
)
\end{lstlisting}


\subsection{Результаты работы}
\begin{lstlisting}
> (tree-similar-p '(1 (2 (3 4)) 5) '((1 2) 3 (4 5)))
T
> (tree-similar-p '(1 (2 (3 4)) 5) '((1 2) 3 (5 4)))
NIL
> (tree-similar-p '((1 2)) '(1 2))
T
> (tree-similar-p '((1) (2)) '((1) 2))
T
> (tree-similar-p '(1 (2 (3))) '(((3) 2 ) 1))
NIL
\end{lstlisting}


\section{Дневник отладки}

\section{Замечания, выводы}
Выполняя эту лабораторную работу, я познакомился с новыми для меня возможностями языка Common Lisp, получил навыки работы со списками. Списки в Lisp'е довольно мощное средство, и в то же время просты в изучении. С их помощью можно удобно представлять деревья, очень важную структуру данных.

\end{document}