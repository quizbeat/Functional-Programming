\documentclass[a4paper, 12pt]{article}

\usepackage{fullpage}
\usepackage{multicol,multirow}
\usepackage{tabularx}
\usepackage{ulem}
\usepackage[utf8]{inputenc}
\usepackage[russian]{babel}
\usepackage{amsmath}
\usepackage{amssymb}
\usepackage{listings}
\usepackage{titlesec}

\titleformat{\section}
  {\normalfont\Large\bfseries}{\thesection.}{0.3em}{}

\titleformat{\subsection}
  {\normalfont\large\bfseries}{\thesubsection.}{0.3em}{}

\titlespacing{\section}{0pt}{*2}{*2}
\titlespacing{\subsection}{0pt}{*1}{*1}
\titlespacing{\subsubsection}{0pt}{*0}{*0}
\usepackage{listings}
\lstloadlanguages{Lisp}
\lstset{extendedchars=false,
	breaklines=true,
	breakatwhitespace=true,
	keepspaces = true,
	tabsize=2
}

\begin{document}

\section*{Отчет по лабораторной работе №\,3
\\по курсу \guillemotleft  Функциональное программирование\guillemotright}

\begin{flushright}
Студент группы 8о-306Б МАИ \textit{Макаров Никита}, \textnumero 11 по списку \\
\makebox[7cm]{Контакты: {\tt quizbeat@gmail.com} \hfill} \\
\makebox[7cm]{Работа выполнена: 17.04.2015 \hfill} \\
\ \\
Преподаватель: Иванов Дмитрий Анатольевич, доц. каф. 806 \\
\makebox[7cm]{Отчет сдан: \hfill} \\
\makebox[7cm]{Итоговая оценка: \hfill} \\
\makebox[7cm]{Подпись преподавателя: \hfill} \\

\end{flushright}


\section{Тема работы}
Обобщённые функции, методы и классы объектов.

\section{Цель работы}
Научиться определять простейшие классы, порождать экземпляры классов, считывать и изменять значения слотов, научиться определять обобщённые функции и методы.

\section{Задание}
Определить обычную функцию {\tt on-single-line-p} - предикат,
\begin{itemize}
\item принимающий в качестве аргумента список точек (радиус-векторов),
\item возвращающий {\tt T}, если все указанные точки лежат на одной прямой (вычислять с допустимым отклонением {\tt tolerance}).
\end{itemize}
Точки могут быть заданы как декартовыми координатами (экземплярами {\tt cart}), так и полярными (экземплярами {\tt polar}).

\begin{lstlisting}
(defvar *tolerance* 0.001)

(defun on-single-line-p (vertices)
  ...)
\end{lstlisting}


\section{Оборудование студента}
Процессор Intel Core i5 2\,@\,1.3GHz, память: 4Gb, разрядность системы: 64.


\section{Программное обеспечение}
ОС Mac OS X 10.9, среда Clozure CL 1.10


\section{Идея, метод, алгоритм}
Воспользуемся уравнением прямой, проходящей через 2 точки. Если 3 точки лежат на одной прямой, то для них выполняется равенство:
$$\frac{x_1 - x_2}{x_3 - x_2} = \frac{y_1 - y_2}{y_3 -y_2}.$$
Чтобы избежать погрешностей при делении, заменим частные на произведения:
$$(x_1 - x_2)*(y_3 - y_2) = (x_3 - x_2)*(y_1 - y_2).$$
Функция {\tt on-single-line-p} рекурсивно проверяет выполнение вышеприведенного равенства для всех элементов списка с помощью функции {\tt on-single-line}, если его длина больше трех. Если длина равна трем, то функция {\tt on-single-line} вызывается один раз. Для длины списка равной двум или одному ответ, очевидно, {\tt T}.

\section{Сценарий выполнения работы}


\section{Распечатка программы и её результаты}

\subsection{Исходный код}
\begin{lstlisting}
(defun square (x) (* x x))

(defclass cart()
    ((x :initarg :x :accessor cart-x)
     (y :initarg :y :accessor cart-y)
    )
)

(defclass polar()
    ((radius :initarg :radius :accessor radius)
     (angle  :initarg :angle  :accessor angle)
    )
)

(defmethod cart-x ((p polar))
    (* (radius p) (cos (angle p)))
)

(defmethod cart-y ((p polar))
    (* (radius p) (sin (angle p)))
)

(defmethod print-object ((c cart) stream)
    (format stream "[CART x:~d y:~d]"
        (cart-x c) (cart-y c)
    )
)

(defmethod print-object ((p polar) stream)
    (format stream "[POLAR radius:~d angle:~d]"
        (radius p) (angle p)
    )
)

(defvar tolerance 0.001)

(defun approx-eq (x y)
    (<= (abs (- x y)) tolerance)
)

(defun on-single-line (v)
    (approx-eq (* (- (cart-x (first v)) (cart-x (second v)))
                  (- (cart-y (third v)) (cart-y (second v))))
               (* (- (cart-x (third v)) (cart-x (second v)))
                  (- (cart-y (first v)) (cart-y (second v))))
    )
)

(defun on-single-line-p (vertices)
    (cond
        ((< (length vertices) 3) T)
        ((= (length vertices) 3) (on-single-line vertices))
        ((> (length vertices) 3)
            (if (on-single-line-p (rest vertices))
                (on-single-line (list (first vertices)
                                      (second vertices)
                                      (third vertices)))
            )
        )
    )
)

(defun t (v) (on-single-line-p v)) ; fast testing

; tests
(defvar p1 (make-instance 'cart :x 1 :y 2))
(defvar p2 (make-instance 'cart :x 2 :y 4))
(defvar p3 (make-instance 'cart :x 3 :y 6))

(defvar p4 (make-instance 'cart :x 0 :y 1))
(defvar p5 (make-instance 'cart :x 1 :y 1))
(defvar p6 (make-instance 'cart :x 2 :y 1))

(defvar p7 (make-instance 'cart :x 4 :y 8.0002))
(defvar p8 (make-instance 'cart :x 5 :y 10.00001))
(defvar p9 (make-instance 'cart :x 5.0005 :y 10.0001))

(defvar p10 (make-instance 'polar :radius 1 :angle 1))
(defvar p11 (make-instance 'polar :radius 2 :angle 1))
(defvar p12 (make-instance 'polar :radius 3 :angle 1))
(defvar p13 (make-instance 'polar :radius 4 :angle 1.000001))

(defvar l1 (list p1 p2 p3)) ; T
(defvar l2 (list p1 p2 p4)) ; NIL
(defvar l3 (list p1 p2 p3 p2 p3 p1)) ; T

(defvar l4 (list p4 p5 p6)) ; T
(defvar l5 (list p7 p8 p9)) ; T
(defvar l6 (list p1 p2 p3 p7 p8 p9)) ; T

(defvar l7 (list p10 p11 p12)) ; T
(defvar l8 (list p10 p11 p12 p13)) ; T
(defvar l9 (list p10 p11 p12 p13 p1)) ; NIL
\end{lstlisting}


\subsection{Результаты работы}
\begin{lstlisting}
> (on-single-line-p l1)
T
> (on-single-line-p l2)
NIL
> (on-single-line-p l3)
T
> (on-single-line-p l4)
T
> (on-single-line-p l5)
T
> (on-single-line-p l6)
T
> (on-single-line-p l7)
T
> (on-single-line-p l8)
T
> (on-single-line-p l9)
NIL
\end{lstlisting}


\section{Дневник отладки}
{\tt 15.05.15} - Исправлен алгоритм.

\section{Замечания, выводы}
С помощью данной лабораторной работы я ознакомился с классами и обобщенными функциями в языке Common Lisp. Синтаксис для создания классов довольно прост, с ним было нетрудно разобраться. Обощенные функции играют немаловажную роль. Они позволяют не создавать несколько похожих функций для разных классов, а просто написать реализации одной функции, которая будет применима для нескольких классов.

\end{document}