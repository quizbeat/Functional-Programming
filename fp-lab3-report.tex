\documentclass[a4paper, 12pt]{article}

\usepackage{fullpage}
\usepackage{multicol,multirow}
\usepackage{tabularx}
\usepackage{ulem}
\usepackage[utf8]{inputenc}
\usepackage[russian]{babel}
\usepackage{amsmath}
\usepackage{amssymb}
\usepackage{listings}
\usepackage{titlesec}

\titleformat{\section}
  {\normalfont\Large\bfseries}{\thesection.}{0.3em}{}

\titleformat{\subsection}
  {\normalfont\large\bfseries}{\thesubsection.}{0.3em}{}

\titlespacing{\section}{0pt}{*2}{*2}
\titlespacing{\subsection}{0pt}{*1}{*1}
\titlespacing{\subsubsection}{0pt}{*0}{*0}
\usepackage{listings}
\lstloadlanguages{Lisp}
\lstset{extendedchars=false,
	breaklines=true,
	breakatwhitespace=true,
	keepspaces = true,
	tabsize=2
}

\begin{document}

\section*{Отчет по лабораторной работе №\,3
\\по курсу \guillemotleft  Функциональное программирование\guillemotright}

\begin{flushright}
Студент группы 8о-306Б МАИ \textit{Макаров Никита}, \textnumero 11 по списку \\
\makebox[7cm]{Контакты: {\tt quizbeat@gmail.com} \hfill} \\
\makebox[7cm]{Работа выполнена: 17.04.2015 \hfill} \\
\ \\
Преподаватель: Иванов Дмитрий Анатольевич, доц. каф. 806 \\
\makebox[7cm]{Отчет сдан: \hfill} \\
\makebox[7cm]{Итоговая оценка: \hfill} \\
\makebox[7cm]{Подпись преподавателя: \hfill} \\

\end{flushright}


\section{Тема работы}
Обобщённые функции, методы и классы объектов.

\section{Цель работы}
Научиться определять простейшие классы, порождать экземпляры классов, считывать и изменять значения слотов, научиться определять обобщённые функции и методы.

\section{Задание}
Определить обычную функцию {\tt on-single-line-p} - предикат,
\begin{itemize}
\item принимающий в качестве аргумента список точек (радиус-векторов),
\item возвращающий {\tt T}, если все указанные точки лежат на одной прямой (вычислять с допустимым отклонением {\tt tolerance}).
\end{itemize}
Точки могут быть заданы как декартовыми координатами (экземплярами {\tt cart}), так и полярными (экземплярами {\tt polar}).

\begin{lstlisting}
(defvar *tolerance* 0.001)

(defun on-single-line-p (vertices)
  ...)
\end{lstlisting}


\section{Оборудование студента}
Процессор Intel Core i5 2\,@\,1.3GHz, память: 4Gb, разрядность системы: 64.


\section{Программное обеспечение}
ОС Mac OS X 10.9, среда Clozure CL 1.10


\section{Идея, метод, алгоритм}
Очевидно, что точки лежат на одной прямой, если их углы в полярном представлении совпадают. Поэтому будем рекурсивно идти по списку точек, преобразовывать точки в полярную систему координат и сравнивать их углы с учетом погрешности {\tt tolerance}. {\tt on-single-line-p} принимает список точек и возвращает {\tt T}, если функция {\tt on-single-line} от этого же списка точек вернула число. {\tt on-single-line} непосредственно сравнивает точки с помощью функции {\tt approx-eq}, и если на каком-либо сравнении разница углов окажется больше {\tt tolerance}, функция будет возвращать вверх по рекурсии {\tt nil}, и следовательно число возвращено не будет.

\section{Сценарий выполнения работы}


\section{Распечатка программы и её результаты}

\subsection{Исходный код}
\begin{lstlisting}
(defun square (x) (* x x))

(defclass cart()
    ((x :initarg :x :accessor cart-x)
     (y :initarg :y :accessor cart-y)
    )
)

(defclass polar()
    ((radius :initarg :radius :accessor radius)
     (angle  :initarg :angle  :accessor angle)
    )
)

(defgeneric get-angle (arg)
    (:documentation "returns angle of arg point")
)

(defmethod get-angle ((c cart))
    (atan (cart-y c) (cart-x c))
)

(defmethod get-angle ((p polar))
    (angle p)
)

(defgeneric get-radius (arg)
    (:documentation "returns radius of arg point")
)

(defmethod get-radius ((c cart))
    (sqrt (+ (square (cart-x c))
             (square (cart-y c)))
    )
)

(defmethod get-radius ((p polar))
    (radius p)
)

(defgeneric to-polar (arg)
    (:documentation "transformation to polar coordinate system")
)

(defmethod to-polar ((p polar)) p)

(defmethod to-polar ((c cart))
    (make-instance 'polar
        :radius (get-radius c)
        :angle (get-angle c)
    )
)

(defmethod print-object ((c cart) stream)
    (format stream "[CART x:~d y:~d]"
        (cart-x c) (cart-y c)
    )
)

(defmethod print-object ((p polar) stream)
    (format stream "[POLAR radius:~d angle:~d]"
        (radius p) (angle p)
    )
)

(defvar tolerance 0.001)

(defun approx-eq (x y)
    (if (and (numberp x) (numberp y))
        (<= (abs (- x y)) tolerance)
        nil
    )
)

(defun on-single-line (vertices)
    (if (= (length vertices) 1)
        (get-angle (first vertices))
        (if (approx-eq (get-angle (first vertices))
                       (on-single-line (rest vertices)))
            (get-angle (first vertices))
            nil
        )
    )
)

(defun on-single-line-p (vertices)
    (numberp (on-single-line vertices))
)

(defvar p1 (make-instance 'cart :x 1 :y 2))
(defvar p2 (make-instance 'cart :x 2 :y 4))
(defvar p3 (to-polar (make-instance 'cart :x 3 :y 6)))
(defvar p4 (make-instance 'cart :x 2 :y 2))
(defvar p5 (make-instance 'polar :radius 42 :angle 1.107111))
(defvar l1 (list p1 p2 p3))
(defvar l2 (list p1 p2 p4))
(defvar l3 (list p1 p2 p3 p5))

\end{lstlisting}


\subsection{Результаты работы}
\begin{lstlisting}
> (on-single-line-p l1)
T
> (on-single-line-p l2)
NIL
> (on-single-line-p l3)
T
\end{lstlisting}


\section{Дневник отладки}

\section{Замечания, выводы}
С помощью данной лабораторной работы я ознакомился с классами и обобщенными функциями в языке Common Lisp. Синтаксис для создания классов довольно прост, с ним было нетрудно разобраться. Обощенные функции играют немаловажную роль. Они позволяют не создавать несколько похожих функций для разных классов, а просто написать реализации одной функции, которая будет применима для нескольких классов.

\end{document}