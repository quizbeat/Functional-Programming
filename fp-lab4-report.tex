\documentclass[a4paper, 12pt]{article}

\usepackage{fullpage}
\usepackage{multicol,multirow}
\usepackage{tabularx}
\usepackage{ulem}
\usepackage[utf8]{inputenc}
\usepackage[russian]{babel}
\usepackage{amsmath}
\usepackage{amssymb}
\usepackage{listings}
\usepackage{titlesec}

\titleformat{\section}
  {\normalfont\Large\bfseries}{\thesection.}{0.3em}{}

\titleformat{\subsection}
  {\normalfont\large\bfseries}{\thesubsection.}{0.3em}{}

\titlespacing{\section}{0pt}{*2}{*2}
\titlespacing{\subsection}{0pt}{*1}{*1}
\titlespacing{\subsubsection}{0pt}{*0}{*0}
\usepackage{listings}
\lstloadlanguages{Lisp}
\lstset{extendedchars=false,
	breaklines=true,
	breakatwhitespace=true,
	keepspaces = true,
	tabsize=2
}

\begin{document}

\section*{Отчет по лабораторной работе №\,4
\\по курсу \guillemotleft  Функциональное программирование\guillemotright}

\begin{flushright}
Студент группы 8о-306Б МАИ \textit{Макаров Никита}, \textnumero 11 по списку \\
\makebox[7cm]{Контакты: {\tt quizbeat@gmail.com} \hfill} \\
\makebox[7cm]{Работа выполнена: 16.05.2015 \hfill} \\
\ \\
Преподаватель: Иванов Дмитрий Анатольевич, доц. каф. 806 \\
\makebox[7cm]{Отчет сдан: \hfill} \\
\makebox[7cm]{Итоговая оценка: \hfill} \\
\makebox[7cm]{Подпись преподавателя: \hfill} \\

\end{flushright}


\section{Тема работы}
Последовательности, массивы и управляющие конструкции Common Lisp.

\section{Цель работы}
Научиться создавать векторы и массивы для представления матриц, освоить общие функции работы с последовательностями, инструкции цикла и нелокального выхода.

\section{Задание}
Запрограммировать на языке Common Lisp функцию, принимающую в качестве единственного аргумента целое число $n$ - порядок матрицы. Функция должна создавать и возвращать двумерный массив, представляющий целочисленную квадратную матрицу порядка {\tt n}, элементами которой являются числа $1, 2, ... n^2$, расположенные по схеме, показанной ниже.

\begin{lstlisting}
(defun matrix-1l-2l (n)
  ...)

(matrix-1l-2l 4) =>
#2A((1  3  4 10)
    (2  5  9 11)
    (6  8 12 15)
    (7 13 14 16))
\end{lstlisting}


\section{Оборудование студента}
Процессор Intel Core i5 2\,@\,1.3GHz, память: 4Gb, разрядность системы: 64.


\section{Программное обеспечение}
ОС Mac OS X 10.9, среда Clozure CL 1.10


\section{Идея, метод, алгоритм}
Можно заметить, что числа в матрице следуют по диагоналям и направление движения меняется на противоположное от диагонали к диагонали. Также можно заметить, что направление движения зависит от длины текущей диагонали: если длина является четным числом, то движение идет вверх, иначе -- вниз. Таким образом, можно построить такую матрицу в два этапа: на первом заполнить побочную диагональ и все, что выше, а на втором этапе оставшуюся часть. Заполнение в два этапа удобно из-за разных условий перехода на следующую диагональ.


\section{Сценарий выполнения работы}


\section{Распечатка программы и её результаты}

\subsection{Исходный код}
\begin{lstlisting}
(defvar matrix (make-array '(1 1))) ; matrix

(defun mprint (m)
  (loop for i below (car (array-dimensions m)) do
        (loop for j below (cadr (array-dimensions m)) do
          (let ((cell (aref m i j)))
            (format t "~a " cell)))
        (format t "~%")))

(defun matrix-1l-2l (n)
    (PROG
        ((i 0) (j 0) (value 1) (last (- n 1))) ; local vars
        (setf matrix (make-array (list n n)))
        (loop for diag-length from 1 to n do
    		(if (evenp diag-length)
    			(let ((end (- diag-length 1)))
    				(loop for l from 0 to end do
    					(setf (aref matrix i j) value)
    					(incf value)
    					(if (/= l end)
                            (do () ((decf i) (incf j)))
    					)
    				)
    				(if (= j last)
    					(incf i)
    					; else
    					(incf j)
    				)
    			)
    			; else
    			(let ((end (- diag-length 1)))
    				(loop for l from 0 to end do
    					(setf (aref matrix i j) value)
    					(incf value)
    					(if (/= l end)
                            (do () ((incf i)(decf j)))
    					)
    				)
    				(if (= i last)
    					(incf j)
    					; else
    					(incf i)
    				)
    			)
    		)
    	)
        (loop for diag-length from last downto 1 do
    		(if (evenp diag-length)
    			(let ((end (- diag-length 1)))
    				(loop for l from 0 to end do
    					(setf (aref matrix i j) value)
    					(incf value)
    					(if (/= l end)
    						(do () ((decf i) (incf j)))
    					)
    				)
    				(incf i)
    			)
    			; else
    			(let ((end (- diag-length 1)))
    				(loop for l from 0 to end do
    					(setf (aref matrix i j) value)
    					(incf value)
    					(if (/= l end)
    						(do () ((incf i) (decf j)))
    					)
    				)
    				(incf j)
    			)
    		)
    	)
    )
    matrix
)

(defun t (x) (mprint (matrix-1l-2l x))) ; fast testing

\end{lstlisting}


\subsection{Результаты работы}
\begin{lstlisting}
> (mprint (matrix-1l-2l 4))
1 3 4 10 
2 5 9 11 
6 8 12 15 
7 13 14 16 
NIL
> (mprint (matrix-1l-2l 5))
1 3 4 10 11 
2 5 9 12 19 
6 8 13 18 20 
7 14 17 21 24 
15 16 22 23 25 
NIL
> (mprint (matrix-1l-2l 6))
1 3 4 10 11 21 
2 5 9 12 20 22 
6 8 13 19 23 30 
7 14 18 24 29 31 
15 17 25 28 32 35 
16 26 27 33 34 36 
NIL
? 
\end{lstlisting}


\section{Дневник отладки}

\section{Замечания, выводы}
Выполняя эту работу, я получил базовые навыки работы с массивами. В ходе изучения возникли некоторые трудности, не сразу получилось создать массив произвольного размера {\tt n}. Еще для меня показался несколько громоздким синтаксис обновления значения массива, но эта проблема решается макросами. Для просмотра матрицы лучше воспользоваться дополнительной функцией печати, которая выводит двумерный массив в удобочитаемом формате.

\end{document}